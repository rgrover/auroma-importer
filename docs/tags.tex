\documentclass[11pt]{article}

%% Things to add:
%%  * Make it extra clear that the input should be plain ASCII. If
%%    anyone uses a special editor like Word, then they should avoid
%%    introducing Word specific formatting.
%%  * Diacritical marks cannot be verified visually until the end of
%%    the import process. This poses a special challenge; people
%%    involved have to be very careful.
%%  * Rename \newpage{} to \pagebreak[]
%%  * Blank lines should be avoided when switching pages; otherwise
%%    they may be mis-interpreted as paragraph endings.
%%  * Allow for comments beginning with %
%%  * Files must be saved in plain text format.
%%  * Add \sc for small caps.
%%  * paragraphs MUST begin on a new line with a \par command
%%  * Talk about blocks within { and }; blocks should not span
%%    paragrpahs (empty lines) unless in a verbatim context.
%%  * nothing spans a paragraph.
%%  * Add tag for Footer text.
%%  * Add tag for Drop.
%%  * The following tags apply to a paragraph; they should be
%%    discussed together: \indent \noindent \center \drop \nodrop
%%    \quote \poem \footer \flushleft \flushright
%%  * diacritical marks need a { } around the character


\newcommand{\cmd}[1]{{\tt $\backslash$#1}}

%% Description of the Language to be used for Tagging
%% ==================================================
%% Rohit Grover <rgrover1@gmail.com>
%% v0.1, February 2011:
%% Draft version.
\title{Description of the Language to be used for Tagging\\(Draft Version)}
\author{Rohit Grover}
\date{Draft V0.1, February, 2011}
\begin{document}
\maketitle

\newpage

\section{Introduction}

\noindent With its intention to archive and publish the works of (and
related to) Sri Aurobindo and The Mother, Aurokruti will be gathering
electronic versions of texts, either through the process of
translating existing materials from other formats, scanning documents,
or even by capturing them manually. In addition to obtaining the plain
texts for the various works, it will be desirable to capture useful
semantic information from the sources so as to bring the quality of
the gathered materials as close to the originals as possible. The
current document presents a simple language consisting of a set of
tags which may be added to the plain texts of the works in order to
enrich their content based on the originals.

The tagged files will subsequently be input to the automated importer
to generate XML (and any other formats as needed for research); and it
is very likely that the final versions will be made available to the
users through a web interface. While having a rich set of tags is a
good thing for describing the originals accurately, it is not our
intention to replicate the works. We would like to keep the tagging
process simple; a simple set of tags will also help to keep our data
processing tool-chain manageable in its complexity.

%% Input Files
%% -----------

%% Each tagged text to be sent to the importer will be a plain ASCII text
%% file containing an entire work. You can create it with any editor. It
%% will contain the text of the document as well as have tags added to
%% highlight special semantic properties.

\section{Input Files}

Each tagged text to be sent to the ``importer'' will be a plain ASCII
text file containing an entire work. It should be possible to create
it with any editor. It will contain the text of the document as well
as have tags added to highlight special semantic properties.

%% Spaces
%% ~~~~~~

%% "Whitespace" characters such as blank or tab are treated uniformly as
%% "space." Several consecutive white space characters are treated as one
%% "space." A single linebreak is treated as a "whitespace."

%% Any empty line between two lines of text defines the end of a
%% paragraph. Several empty lines are treated the same as one empty
%% line. A new paragraph can also be begun with a +\par+ tag--this is the
%% recommended approach.

\subsection{Spaces}

``Whitespace" characters in the text, such as blanks or tabs, will be
treated uniformly as ``space." Several consecutive white space
characters will be treated the same as one single "space." A single
linebreak is treated as a ``whitespace."

Any empty line between two lines of text defines the end of a
paragraph. Several empty lines are treated the same as one empty
line. A new paragraph can also be begun with a {\tt $\backslash$par}
tag--this is the recommended approach.

%% Special Characters
%% ~~~~~~~~~~~~~~~~~~

%% The following symbols are reserved characters that have special
%% meaning:

%% _______________________
%% \ { }
%% _______________________

%% As you will see, these characters can be used in your documents all
%% the same by adding a prefix backslash:


%% ----------------------
%% \\ \{ \}
%% ----------------------

%% Unusual symbols can described either by using diacritical marks (for
%% accents) or using Unicode Character-points directly (such as \Uxxxx,
%% where +xxxx+ is the UTF-16 representation of a character).

\newpage

\subsection{Special Characters}

The following symbols are reserved characters that have special
meaning:

\begin{quote}
$\backslash$ \{ \}
\end{quote}

\noindent As you will see, these characters can be used in your documents
by adding a backslash prefix:

\begin{quote}
$\backslash$$\backslash$ $\backslash$\{ $\backslash$\}
\end{quote}


%% Tags
%% ~~~~

%% Tags and commands are case sensitive. They usually start with a
%% backslash \ and then have a name consisiting of letters only. Command
%% names are terminated by a space, a number, or any other punctuation
%% mark.

%% Some commands need a parameter which has to be given between curly
%% braces { } after the command name; for example:

%% .Examples of Using Tags--even ones which take arguments
%% ===================================
%% Please, start a new line right here!\newline Thank you!

%% The above should place 'Thank you!' on a separate line\footnote{this
%% is a footnote appearing right after the word 'line'}.
%% ===================================


%% In the above tagged text, we used the \newline command to force a new
%% line. We also used the \footnote tag, which takes an argument within
%% curly braces.

\subsection{Tags}

Tags and commands are case sensitive. They usually start with a
backslash $\backslash$ and then have a name consisiting of letters
only; such as \cmd{newline}. In some cases, tags can take an argument,
such as: \cmd{footnote\{some text\}}\ ---such tags are called
commands. Tags and commands may be terminated by a space, a number, or
any other punctuation mark. The following is an example of the use of
tags and commands in text:
\begin{quote}
Please, start a new line right here!\cmd{newline} Thank you!

The above should place 'Thank you!' on a separate line\cmd{footnote\{this
is a footnote appearing right after the word 'line'\}}.
\end{quote}

In the above tagged text, we used the \cmd{newline} tag to force a new
line. We also used the \cmd{footnote} command, which takes an argument
within curly braces.



%% Input File Structure
%% ~~~~~~~~~~~~~~~~~~~~

%% When the importer processes an input file, it expects the file to
%% follow a certain structure.

%% At the head of an input file, there can be an optional header
%% containing meta-data information in the form of the \title, \tagger,
%% and \date tags. The following is an example:


%% -----------------------------
%% \title{The Life Divine}
%% \tagger{Dr. Goodman}
%% \date{12th February, 2011}

%% ...the tagged text
%% ...
%% Simple is {\it beaufitul}.
%% ...
%% -----------------------------

%% Thereafter, the body of the tagged text follows.

\subsection{Input File Structure}

When the importer processes an input file, it expects the file to
follow a certain structure.

At the head of an input file, there can be an optional header
containing meta-data information in the form of the \cmd{title},
\cmd{tagger}, and \cmd{date} tags. The following is an example:

\begin{quote}
  \cmd{title\{The Life Divine\}}\\
  \cmd{tagger\{Dr. Goodman\}}\\
  \cmd{date\{12th February, 2011\}}\\
\\
  ...the tagged text \\
  Simple is \{\cmd{it} beaufitul\}.\\
\end{quote}

%% Tagging Text
%% ------------

\section{Tagging Text}

%% Paragraphs
%% ~~~~~~~~~~
\subsection{Paragraphs}

%% The most important text unit in a tagged document is the paragraph. We
%% call it a "text unit" because a paragraph is the typographical form
%% which should reflect one coherent thought, or one idea. If a new
%% thought begins, a new paragraph should begin, and if not, only
%% linebreaks should be used.

%% Paragraphs should be tagged as beginning with a +\par+, or a
%% +\firstpar+ in the case where the paragraph is the first within its
%% parent container (such as the first paragraph of a chapter).

%% Although an empty line also marks the end of a paragraph, it is better
%% to be explicit about it through the use of +\par+ at the beginning of
%% a paragraph. If an empty line is used to separate paragraphs, a +\par+
%% is assumed by default, instead of a +\firstpar+.

The most important text unit in a tagged document is the paragraph. We
call it a ``text unit" because a paragraph is the typographical form
which should reflect one coherent thought, or one idea. If a new
thought begins, a new paragraph should begin, and if not, only
linebreaks should be used.

Paragraphs should be tagged as beginning wihh a \cmd{par}, or a
\cmd{firstpar} in the case where the paragraph is the first within its
parent container (such as the first paragraph of a chapter).

Although an empty line also marks the end of a paragraph, it is better
to be explicit about it through the use of \cmd{par} at the beginning
of a paragraph. If an empty line is used to separate paragraphs, a
\cmd{par} is assumed by default, instead of a \cmd{firstpar}.

%% Paragraph Containers
%% ~~~~~~~~~~~~~~~~~~~~

\subsection{Paragraph Containers}

Paragraphs of a text may either stand independently, or, as is
commonly the case, they may be structured logically into a higherarchy
of levels: chapters, book-parts, and so on.

We shall not assume or adopt any single hierarchy of these groupings
since such a choice cannot not be consistent across the many works we
wish to encode; but rather we will limit ourselves to a logical scheme
for these containers and permit the tagger to select the appropriate
categorization in each case--the table-of-contents for a work should
suggest this hierarchy quite simply in most cases.

In order to simplify the task of expressing the container hierarchy
without getting caught up in the details of levels and level-numbers,
only the following tags are made available to the tagger.

\begin{quote}
  \cmd{hn} Heading Number\\
  \cmd{ht} Heading Title\\
  \cmd{push}\\
  \cmd{pop}\\
\end{quote}

Any text is assumed to begin with the current container level set to
the highest possible logical value---so for instance, we know that
``The Life Divine" begins with ``Book One", and so the tagger can
assume that at the beginning of the text the current container level
is that of a ``Book". The tagger need not specify this level of
detail---that the starting container is a Book,---the information
required to be extracted from the tagged text doesn't depend on this
kind of detail.

At any time, a new container can be introduced (at the current level)
by using either of the following naming tags: \cmd{hn} (which
introduces a container through a heading-number) or \cmd{ht} (which
introduces a heading-title). When used in conjunction, \cmd{ht} and
\cmd{ht} describe the same container heading.

So, for instance, the following is one way to introduce Book One in
the Life Divine:

\begin{quote}
  \cmd{par}\\
  \cmd{hn} BOOK ONE\\
  \cmd{ht} Omnipresent Reality \cmd{newline} And \cmd{newlilne} The Universe
\end{quote}

\noindent Continuing with this example, we can descend from the level of a Book
into Chapters using the following tag:

\begin{quote}
  \cmd{push}
\end{quote}

\noindent Thereafter, ``Chapter I" within the book can be tagged as follows:

\begin{quote}
  \cmd{par}\\
  \cmd{hn} Chapter I\\
  \cmd{ht} The Human Aspiration
\end{quote}

\noindent If only one of a heading-number or heading-title is
specified, that still introduces a new container at the current
level--usually, both the number and the title are available.

When all the containers at the current level have been
exhausted, it is possible to jump back to the parent level using the
following tag:

\begin{quote}
  \cmd{pop}
\end{quote}

Using this simple scheme, it should be possible to define any kind of
hierarchy to an arbitrary level of depth. Note that there should be as
many occurrences of \cmd{pop} as there are for \cmd{push}; i.e. the
document should end up with the current container level being at the
top.

%% Linebreaking and Pagebreaking
%% ~~~~~~~~~~~~~~~~~~~~~~~~~~~~~
\subsection{Linebreaking and Pagebreaking}


%% Often books are typeset with each line having the same length. For our
%% purposes, however, since the content will most likely be displayed on
%% a web browser, we should not depend on assumptions about line
%% widths. Pagebreaks, however, are still important--we would like to
%% retain this information so that the users can easily switch between
%% electronic and hard-copy versions of a book.

%% The manner in which the paragraphs are eventually typeset on the
%% display depends on the current container level and the attributes of
%% the paragraph--and the tagger should mostly remain isolated from these
%% details.

%% Normally the first line of every paragraph is indented by default
%% (except the first paragraph within a container). If the indenting for
%% a paragraph is to be dropped, then a +\noindent+ should be used
%% immediately following the +\par+; similarly if indenting is required
%% it may be explicitly requested using a +\indent+.

%% In special cases, it might be necessary to order a linebreak--as is
%% often necessary when defining container headings. For these cases,
%% +\newline+ starts a new line without starting a new
%% paragraph. Similarly, +\newpage{pagenumber}+ starts a new page. With
%% +\newpage+, the pagenumber within the square brackets is an optional
%% argument. It can be left empty if there is no pagenumber to be
%% displayed; otherwise it should be specified.


Often books are typeset with each line having the same length. For our
purposes, however, since the content will most likely be displayed on
a web browser, we should not depend on assumptions about line
widths. Pagebreaks, however, are still important---we would like to
retain this information so that the users can easily switch between
electronic and hard-copy versions of a book.

The manner in which the paragraphs are eventually typeset on the
display depends on the current container level and the attributes of
the paragraph---and the tagger should mostly remain isolated from these
details.

Normally the first line of every paragraph is indented by default
(except the first paragraph within a container). If the indenting for
a paragraph is to be dropped, then a \cmd{noindent} should be used
immediately following the \cmd{par}; similarly if indenting is
required it may be explicitly requested using a \cmd{indent}.

In special cases, it might be necessary to order a linebreak--as is
often necessary when defining container headings. For these cases,
\cmd{newline} starts a new line without starting a new
paragraph. Similarly, \cmd{newpage\{pagenumber\}} starts a new
page. With \cmd{newpage}, the pagenumber within the curly brackets is
an optional argument. It can be left empty if there is no pagenumber
to be displayed; otherwise it should be specified.

%% Special Characters and Symbols
%% ~~~~~~~~~~~~~~~~~~~~~~~~~~~~~~
\subsection{Special Characters and Symbols}

%% Quotation Marks
%% ^^^^^^^^^^^^^^^
\subsubsection{Quotation Marks}

You may use the " character for quotation marks as you would on a
typewriter, but there are better alternatives. In publishing there are
special opening and closing quotation marks. Use two `s (grave
accents) for opening quotation marks and two 's (apostrophe) for
closing quotation marks. For single quotes, you use just one of each.

\begin{quote}
\begin{verbatim}
  Here is an example of how you would ``quote''.
\end{verbatim}
\end{quote}

%% Dashes and Hyphens
%% ^^^^^^^^^^^^^^^^^^
\subsubsection{Dashes and Hyphens}


%% You can use three kinds of dashes. You can use these with different
%% numbers of consecutive dashes. The names for these dashes are: '-'
%% hyphen, '--' en-dash, and '---' em-dash. An en-dash is used to obtain
%% a dash the same width as an 'n'. Likewise, an em-dash gives a dash as
%% wide as the letter 'm'. A hyphen is used to separate word-parts or to
%% specify number ranges.

You can use three kinds of dashes. You can use these with different
numbers of consecutive dashes. The names for these dashes are: `{\tt
  -}' hyphen, `{\tt --}' en-dash, and `{\tt ---}' em-dash. An en-dash
is used to obtain a dash the same width as an 'n'. Likewise, an
em-dash gives a dash as wide as the letter 'm'. A hyphen is used to
separate word-parts or to specify number ranges.


%% Elipsis (...)
%% ^^^^^^^^^^^^^^

\subsubsection{Elipsis (\ldots)}

On a typewriter, a comma or a period takes the same amount of space as
any other letter. In publishing, these characters occupy only a little
space and are set very close to the preceding letter. Therefore you
cannot enter `ellipsis' by just typing three dots, as the spacing
would be wrong. There is a special command for these dots, it is
called \cmd{dots}.

%% Accents and Special Characters
%% ^^^^^^^^^^^^^^^^^^^^^^^^^^^^^^
\subsubsection{Accents and Special Characters}

We support the use of accents and special characters from many
languages. The following table shows all accents being applied to the
letter {\tt o}. Naturally, other letters work too.

%% [width="15%"]
%% |===============================
%% |\=o | latexmath:[$\text{\={o}}$]
%% |\.o | latexmath:[$\text{\.{o}}$]
%% |\,o | latexmath:[$\text{\,{o}}$]
%% |\'o | latexmath:[$\text{\'{o}}$]
%% |\"o | latexmath:[$\text{\"{o}}$]
%% |\^o | latexmath:[$\text{\^{o}}$]
%% |\`o | latexmath:[$\text{\`{o}}$]
%% |\~o | latexmath:[$\text{\~{o}}$]
%% |\co | latexmath:[$\text{\c{o}}$]
%% |\do | latexmath:[$\text{\d{o}}$]
%% |===============================

\vspace{0.4cm}
\begin{tabular}{c|c}
  Sequence & Produces\\
\hline
  \cmd{=o} & \=o \\
  \cmd{.o} & \.o \\
  \cmd{'o} & \'o \\
  \cmd{"o} & \"o \\
  \cmd{\textasciicircum o} & \^o \\
  \cmd{`o} & \`o \\
  \cmd{\textasciitilde o} & \~o \\
  \cmd{co} & \c{o}\\
  \cmd{,o} & \d{o}\\
  \cmd{do} & \d{o}\\
\end{tabular}
\vspace{0.4cm}

If you need to use characters outside the Latin alphabet, you can
enter Unicode Codepoints directly. To do this, use a sequence such as
\cmd{uxxxx} where the {\tt xxxx} stands for the 4 digit hexadecimal
codepoint. So, for instance, the character \~o, which is represented
by the codepoint {\tt 0x00F5}, can be written anywhere in the document
as \cmd{u00f5}.


%% Footnotes
%% ~~~~~~~~~

%% With the command +\footnote{footnote text}+ a footnote is printed at
%% the foot of the current page. Footnotes should always be put after the
%% word or sentence they refer to. Footnotes referring to a sentence or a
%% part of it should therefore be put after the comma or period.

\subsubsection{Footnotes}

With the command \cmd{footnote\{footnote text\}} a footnote is printed
at the foot of the current page. Footnotes should always be put after
the word or sentence they refer to. Footnotes referring to a sentence
or a part of it sould therefore be put after the comma or period.

%% Emphasized Words
%% ~~~~~~~~~~~~~~~~

%% Words or ranges of words may be emphasized by typesetting them in an
%% italic font. We provide the command +\emph{text}+ to emphasize
%% text. In addition to this command, we also provide the +\it+ tag which
%% sets-up italics for the text which follows in the current paragraph or
%% the current block (within the surrounding \{ and \}).

\subsubsection{Emphasized Words}

Words or ranges of words may be emphasized by typesetting them in an
italic font. We provide the tag \cmd{it} to emphasize text. It sets-up
italics for the following text in the current paragraph or the current
block--i.e. until the end of the surrounding \{ and \} block.

\begin{quote}
  The use \{\cmd{it} hello world\} will setup the text ``hello world''
  in italics.
\end{quote}


%% Bold Face
%% ~~~~~~~~~

%% Similar to emphasizing text with italics, we provide +\bf{text}+ for bold face
%% text.


\subsubsection{Bold Face}

Similar to emphasizing text with italics, we provide the \cmd{bf} tag
for bold face text.  It sets-up bold-face font for the following text
in the current paragraph or the current block--i.e. until the end of
the surrounding \{ and \} block.


\subsubsection{Quotation Following a Heading}

The tag \cmd{hquote} should be used to mark the current paragraph as
containing a quotation which immediately follows a chapter (or
high-level container) heading. So, the following is the tagged version
of the quote which follows Chapter I of Book One of ``The Life Divine'':

\begin{quote}
  \cmd{par}\cmd{hquote} She follows to the goal of those that are
  passing on beyond, she is the first in the eternal succession of the
  dawns that are coming,{\tt ---}Usha widens bringing out that which lives,
  awakening someone who was dead. \cmd{dots} What is her scope when she
  harmonises with the dawns that shone out before and those that now
  must shine? She desires the ancient mornings and fulfils their
  light; projecting forwards her illumination she enters into
  communion with the rest that are to come.
\end{quote}

%% Quotation References
%% ~~~~~~~~~~~~~~~~~~~~

%% We provide the command +\ref{text}+ for adding references to
%% quotations.

\subsubsection{References for \cmd{hquote}}

We provide the command \cmd{hquoteref\{text\}} for adding references
for \cmd{hquote} paragraphs. So, the following is the tagged version
of the second quote which follows Chapter 1 of Book 1 of ``The Life
Divine'':

\begin{quote}
  \cmd{par}\cmd{hquote} Threefold are those supreme births of this
  divine force that is in the world, they are true, they are
  desirable; he moves there wide-overt within the Infinite and shines
  pure, luminous and fulfilling. \cmd{dots} That which is immortal in
  mortals and possessed of the truth, is a god and established
  inwardly as an energy working out in our divine powers. \cmd{dots}
  Become high-uplifted, O Strength, pierce all veils, manifest in us
  the things of the Godhead.
  \cmd{hquoteref}\{\cmd{it} Vamadeva{\tt ---}Rig Veda.\}
\end{quote}

\subsubsection{Quotation and Verse Paragraphs}

The tag \cmd{quote} should be used to mark the current paragraph as
containing a quotation, an important phrase, or an example. Quotation
paragraphs are typically indented as a whole. The \cmd{verse} tag is
useful for poems where preserving the linebreaks is important.

%% Itemize and Enumerate
%% ~~~~~~~~~~~~~~~~~~~~~

%% The +\itemize+ tag is suitable for simple lists (using bullets). It
%% applies to the current paragraph.

%% +\enumerate{number}+ generates lists with numbers as supplied by the tagger.

\subsubsection{Itemize and Enumerate}

The \cmd{itemize} tag is suitable for simple lists (using
bullets). \cmd{enumerate\{number\}} generates lists with numbers as
supplied by the tagger. They apply to the current paragraph.

%% Alignment
%% ~~~~~~~~~

%% The tags +\flushleft+ and +\flushright+ generate paragraphs which are
%% either left- or right-aligned. The +\center+ tag generated centered
%% text. These tags apply to the current paragraph.

\subsubsection{Alignment}

The tags \cmd{flushleft} and \cmd{flushright} generate paragraphs
which are either left- or right-aligned. The \cmd{center} tag
generated centered text. These tags apply to the current paragraph.

%% Printing Verbatim
%% ~~~~~~~~~~~~~~~~~

%% Paragraphs containing the +\verbatim+ tag will have their text taken
%% literally, as if they were typed on a typewriter, with all the
%% linebreaks and spaces, without any other formatting. This tag can
%% prove to be a useful fall back for the cases where supported tags are
%% unable to describe text as it appears in the original work.

\subsubsection{Printing Verbatim}

The command \cmd{verbatim\{\}} will have its text taken literally, as
if it was typed on a typewriter, with all the linebreaks and spaces,
without any other formatting. This tag can prove to be a useful
fall-back for the cases where supported tags are unable to describe
text as it appears in the original work. The following is an example:

\begin{quote}
  \cmd{par}\cmd{verbatim}\{
\begin{verbatim}
Text with special formating    ..    right-justified text

Author's signature.
\end{verbatim}
\}
\end{quote}

\noindent Note that linebreaks are preseved within the verbatim
command. Verbatim, if used, should be the only tag within the
containing paragraph.

\end{document}
